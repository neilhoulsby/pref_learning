
\documentclass[final,hyperref={pdfpagelabels=false}]{beamer}
%\usepackage{grffile}
\mode<presentation>{\usetheme{ktj}}
\usepackage[english]{babel}
\usepackage{graphicx}
\usepackage{epstopdf}
\usepackage{subfigure}
\usepackage{asymptote}
\usepackage{cancel}
\usepackage[latin1]{inputenc}
\usepackage{asymptote}

\usepackage{mathrsfs, amsmath, amssymb, latexsym}
\boldmath
\usepackage[orientation=portrait,size=a1,scale=0.9,debug]{beamerposter}
% change list indention level
% \setdefaultleftmargin{3em}{}{}{}{}{}

%\usepackage{snapshot} % will write a .dep file with all dependencies, allows for easy bundling

\newcommand{\pphantom}{\textcolor{ta3aluminium}} % phantom introduces a vertical space in p formatted table columns??!!

\newcommand{\met}{\cancel{E}_T}
\newcommand{\mpt}{\cancel{p}_T}
\newcommand{\vmpt}{\vec{\cancel{p}}_T}
\newcommand{\meff}{m_\text{eff}}
\newcommand{\mct}{m_{CT}}
\newcommand{\mtt}{m_{T2}}
\newcommand{\jmdp}{\Delta\phi_{\text{jet}/\met}}
\newcommand{\ttbar}{t\,\overline{t}}
\newcommand{\nexp}[1]{\langle #1 \rangle}
\newcommand{\lumi}{\mathcal{L}}
\newcommand{\squark}{\tilde{q}}
\newcommand{\gluino}{\tilde{g}}
\newcommand{\lsp}{\chi^0_1}
\newcommand{\bigo}{\mathcal{O}}
\newcommand{\inv}[1]{$\text{#1}^{-1}$}

\listfiles

%%%%%%%%%%%%%%%%%%%%%%%%%%%%%%%%%%%%%%%%%%%%%%%%%%%%%%%%%%%%%%%%%%%%%%%%%%%%%%%%%%%%%%
%\graphicspath{ {/Users/tengman/Desktop/Plots/} }
 
\title{{\huge Where could SUSY be hiding?} \\ \hskip13ex -- Searching for di-slepton production in ATLAS}
\author{Sarah Williams\\Supervised by Dr. Christopher Lester}
\institute{High Energy Physics Group, Cavendish Laboratory}
\date{}

%%%%%%%%%%%%%%%%%%%%%%%%%%%%%%%%%%%%%%%%%%%%%%%%%%%%%%%%%%%%%%%%%%%%%%%%%%%%%%%%%%%%%%
\newlength{\columnheight}
\setlength{\columnheight}{105cm}


%%%%%%%%%%%%%%%%%%%%%%%%%%%%%%%%%%%%%%%%%%%%%%%%%%%%%%%%%%%%%%%%%%%%%%%%%%%%%%%%%%%%%%
\begin{document}
\begin{frame}
  \vskip2ex
  \begin{columns}
    % ---------------------------------------------------------%
    % Set up a column 
    \begin{column}{.49\textwidth}
      \begin{beamercolorbox}[center,wd=\textwidth]{postercolumn}
        \begin{minipage}[T]{.95\textwidth}  % tweaks the width, makes a new \textwidth
          \parbox[t][\columnheight]{\textwidth}{ % must be some better way to set the the height, width and textwidth simultaneously
            % Since all columns are the same length, it is all nice and tidy.  You have to get the height empirically
            % ---------------------------------------------------------%
            % fill each column with content
            
            \begin{block}{Introduction to SUSY}
            \begin{itemize}
            \item Supersymmetry is an extension to the Standard Model that relates fermions and bosons.
            \item Because no SUSY particles have yet been observed it must be a broken symmetry, and thus predicts heavier superpartners for all of the Standard Model particles.
            \item The MSSM ("Minimal Supersymmetric Standard Model") has minimal particle content- the superpartners are divided into squarks , gluinos, neutralinos , charginos and sleptons.
            \item In R-parity conserving SUSY, the lightest neutralino is the lightest supersymmetric particle (LSP)- and is stable.
            \end{itemize}
      	\begin{itemize}
            \item The supersymmetric partners have the same weak and strong couplings as their Standard Model counterparts.
            \item Consequently, being a hadron collider, the production of squarks and gluinos is favoured at the LHC as they are produced by the strong interaction. 
            \item Typical signatures include large missing energy, and high $p_{T}$ jets and leptons.
	   \end{itemize}
            \end{block}
           
            \vskip2ex

	\begin{block}{Motivation for searching for slepton production}
	
	\begin{itemize}
	\item The SUSY searches in ATLAS have been very successful this year at excluding large areas of SUSY parameter space.
	\item As an example in mSUGRA models with $tan(\beta)=10, A_{0}=0$ and $\mu>0$ the 0-lepton search excluded squarks and gluinos of equal mass below $950GeV$ with just $165fb^{-1}$ of data. (See ATLAS-CONF-2011-086)
	\item These searches are very powerful at detecting (reasonably) light squarks and gluinos, but are not very sensitive to models with heavy squarks and gluinos $\rightarrow$ For these models, direct production of the lighter gauginos and sleptons can be important
	\item The low masses can (partly) counteract the small cross sections for direct production:
	
         \begin{figure}[htbp]
               
               \caption{Feynman diagram for di-slepton production at the LHC}
	      \label{fig:example}

	 \end{figure}
	\item In mSUGRA the slepton/squark mass relation means if there is slepton production there should also be significant squark production- mass gap between squarks and sleptons does not compensate for the low Electroweak cross section- so to observe sleptons we should also have observed squarks and gluinos...
	\item But...assuming that SUSY is hiding somewhere....there are theories where slepton production could be a discovery channel.
	\end{itemize}
	
	\end{block}
	
	
	\vskip2ex
	
	\begin{block}{Where to look?}
	\begin{itemize}
	\item Such a search would be sensitive to any new particles with weak couplings carrying lepton number, which then decay to a lepton and an invisible particle.
	\item For a framework use the pMSSM ("Phenomenological MSSM")- this applies several phenomenological constraints to the unconstrained MSSM which has 105 free parameters.The pMSSM then requires only 19 input parameters which include the sfermion masses.
	\item Look for models with mass hierarchy $m_{\tilde{\chi}}, m_{\tilde{q}}>m_{\tilde{l}}>m_{\tilde{\chi}_{1}^{0}}$ which provides a competitive search for SUSY if $m_{\tilde{g}},m_{\tilde{q}}>>m_{\tilde{l}}$.
	\item Mass spectrum for such a model shown below:
	\begin{center}
	    \begin{figure}[htbp]
               
               \caption{SUSY mass spectrum for a pMSSM model where di-slepton production could give a signal}
	      \label{fig:example}

	 \end{figure}
	\end{center}

	
	
	\end{itemize}
	
	
	
	\end{block}
	
	
            }
        \end{minipage}
      \end{beamercolorbox}
    \end{column}
    % ---------------------------------------------------------%
    % end the column

    % ---------------------------------------------------------%
    % Set up a column 
    \begin{column}{.49\textwidth}
      \begin{beamercolorbox}[center,wd=\textwidth]{postercolumn}
        \begin{minipage}[T]{.95\textwidth} % tweaks the width, makes a new \textwidth
          \parbox[t][\columnheight]{\textwidth}{ % must be some better way to set the the height, width and textwidth simultaneously
            % Since all columns are the same length, it is all nice and tidy.  You have to get the height empirically
            % ---------------------------------------------------------%
            % fill each column with content
            
            		
            \begin{block}{The Di-Slepton Signature}
            \begin{itemize} 
            \item Feynman diagram for di-slepton production shown below:
	   \begin{figure}[htbp]
	      \centering
	      \caption{Feynman diagram showing the production of a slepton pair, where each of the sleptons decays to a lepton and an invisible neutralino (the lsp)}
	      \label{fig:sdecay}
	   \end{figure}
        	   \item Characteristics of such an event are two high $p_{T}$ isolated, opposite signed, same flavour leptons, no jets except for ISR and pileup, and missing transverse energy ($E_{T}^{miss}$)
	   \item Major backgrounds come from di-leptonic $t\bar{t}$, diboson and $Z\rightarrow\tau\tau\rightarrow ee/\mu\mu$ events, as well as other backgrounds with fake leptons and/or missing energy (single top, QCD...)
	   
	          \end{itemize}
            \end{block}
            \vskip2ex

            
            \begin{block}{Opposite Sign Di-lepton SUSY searches in ATLAS}
            
            \begin{itemize}
            \item SUSY searches in events in ATLAS with 2 oppositely signed leptons and missing energy are sensitive to events involving cascade decays:
   
              \begin{figure}[htbp]
		     \centering

               
               \caption{Feynman diagram showing the cascade decay of a squark where two oppositely signed same flavour leptons are produced.}
	      \label{fig:example}
     
		     \end{figure}
		     
            \item Plots below show kinematic distributions after event selection, object selection and event cleaning, and with a requirement of exactly two leptons, for $165pb^{-1}$ data.

		  \begin{figure}[htbp]
		     \centering
	
		  \end{figure} \vskip-2ex

		  \begin{figure}[htbp]
		     \centering

		      \caption{Data vs MC distributions for 2-lepton events in the ATLAS detector for $165pb^{-1}$ of data}
		  \end{figure} \vskip-2ex
		  
		  
		 \end{itemize}
		



            \end{block}
            \vskip2ex
            
            \begin{block}{Chasing down a signal...}
            \begin{itemize}
           \item Need to define an approach for searching for di-slepton production in 2 lepton events... current ideas involve:
		              \end{itemize}
		                 \begin{enumerate}
            \item Central Jet Veto- this would reduce $t\bar{t}$ background.
            \item Potentially use the "Stransverse mass" variable $m_{T2}$  [C.G. Lester and D. J. Summers, 1999], defined as:
	$m_{T2}= \min_{\vec{\tilde{p}}_T + \vec{\tilde{q}}_T \;=\; \vmpt}
				 \max\left( m_T [a^\alpha_T, \tilde{p}^\alpha_T(\tilde{\chi})], 
				 		m_T [b^\alpha_T, \tilde{q}^\alpha_T(\tilde{\chi})] \right)$
	where here \textbf{a} and \textbf{b} refer to the transverse momenta of the leptons and \textbf{p} and \textbf{q} to the hypothetical momenta of the neutralinos. It can be shown to be bounded above by the mass of the pair-produced parents.
	\item Investigating possible angular variables.
	
            
\end{enumerate}
\begin{itemize}
\item Hopefully there will be more to come in the future....
\end{itemize}
		                          \end{block}
            
            
            
           \vskip2ex


           }        \end{minipage}
      \end{beamercolorbox}
    \end{column}
    % ---------------------------------------------------------%
    % end the column
  \end{columns}
  \vskip1ex
\end{frame}
\end{document}


%%%%%%%%%%%%%%%%%%%%%%%%%%%%%%%%%%%%%%%%%%%%%%%%%%%%%%%%%%%%%%%%%%%%%%%%%%%%%%%%%%%%%%%%%%%%%%%%%%%%
%%% Local Variables: 
%%% mode: latex
%%% TeX-PDF-mode: t
%%% End:
